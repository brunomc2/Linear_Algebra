% Fonte tamanho 11, duas colunas, tipo artigo.
\documentclass[11pt, twocolumn]{article}

% Escrita em português
\usepackage[utf8]{inputenc}
\usepackage[T1]{fontenc}
\usepackage[portuges, english]{babel}

% Inclusão de Imagens
\usepackage{graphicx}

% Correção de Margens
\usepackage
[
        top = 2.5cm,
        left = 2.5cm,
        right = 2.5cm,
        bottom = 2.5cm
]{geometry}

% Símbolos Matemáticos
\usepackage{amsmath}
\usepackage{mathrsfs}
% Altera sin(x) para sen(x)
\def\sen{\mathop{\mbox{\normalfont sen}}\nolimits}

% Unidades do SI
\usepackage{siunitx}

% Tabelas
\usepackage{booktabs}
\usepackage{array}
\usepackage{multirow}

% Bibliografia
\usepackage
[
        style     = ieee,
        citestyle = numeric
]{biblatex}
\bibliography{bibliography.bib}


% Pacotes de rascunho
\usepackage{lipsum}

% Variáveis de informação
% Título
\newcommand{\vartitulo}{Título do Relatório}
% Nome (Formato A. B. C. Sobrenome)
\newcommand{\varautor}{A. B. C. Sobrenome}
% Instituição de Ensino
\newcommand{\varinstituicao}{Universidade Universidade}
% Departamento
\newcommand{\vardepartamento}{Departamento de Departamento}
% Informações de contato
\newcommand{\varcontato}{Caixa Postal: XXXXX -
                         CEP: XXXXX-XXX -
                         Cidade -
                         Estado \\
                         Fone: +XX-XX-XXXX-XXXX -
                         Fax: +XX-XX-XXXX-XXXX \\
                         email@de.contato}

% Informações para o título
\title{\vartitulo}
\author
{
        \varautor \\
        \varinstituicao \\
        \vardepartamento \\
        \varcontato
}
\date{}

\begin{document}
        \selectlanguage{portuges}
        \sisetup{output-decimal-marker = {,}}
	\twocolumn
        [
% Informações autorais e de contato
                \maketitle

% Resumo do trabalho
                \abstract
                {
                        % Coloque aqui o resumo do trabalho
                        \lipsum[1] Vult.
                        \newline

                        % Palavras Chave
                        Palavras-chave: Alguma, coisa, deve ser,

                        % Espaço livre
                        \vspace{2\baselineskip}
                }
        ]



% Objetivo
        \section{Objetivo}
		Nesse trabalho, por meio do Método de Mínimos
		Quadrados, determinaram-se os pesos para o 
		ajuste das notas de Física Básica Experimental 
		2, de modo a maximizar, na média, as notas de 
		todos os alunos.

% Fundamentação Teórica
        \section{Fundamentação Teórica}
		As notas da matéria se dividem em relatórios e 
		provas. Até o momento de escrita desse trabalho,
		dispunha-mos das notas de apenas dois dos três 
		relatórios e provas realizados para o semestre,
		contudo, estabelecida a validade e a praticidade
		do seguinte método para tal aplicação como 
		forma de 
		\begin{otherlanguage}{english}
			\textit{proof of concept}
		\end{otherlanguage}
		pode-se então aplicá-lo, mantida a 
		infraestrutura computacional, de maneira ágil 
		para a coleção completa das notas do semestre.

		A nota é calculada através de uma média 
		ponderada simples, isto é, se as notas dos 
		relatórios são $R_1$ e $R_2$, com pesos 
		$P_{R1}$ e $P_{R2}$, respectivamente, e, para 
		a prova, têm-se as notas $P_1$ e $P_2$
		com pesos $N_{P1}$ e $N_{P2}$, a nota final 
		$N_F$ pode ser dada por
                \begin{equation}
			\frac{P_{N1}N_1 + P_{N2}N_2 + 
			      P_{R1}R_1 + P_{R2}R_2}
			     {P_{N1} + P_{N2} + P_{R1} + P_{R2}} 
                \end{equation}
		e, deseja-se tentar aproximar, para todos os 
		alunos, a nota $N_F$ de $10$. Isto é, haverá 
		um sistema com linhas do tipo 
                \begin{equation}
			P_{N1}N_1 + P_{N2}N_2 + 
			P_{R1}R_1 + P_{R2}R_2
			= 10 \cdot
			\left(
			P_{N1} + P_{N2} + P_{R1} + P_{R2} 
			\right)
                \end{equation}
		

                \begin{equation*}
                        s = \sigma_{\overline{x}} * \sqrt{n}
                \end{equation*}


% Procedimento Experimental
        \section{Procedimento Experimental}
                \lipsum[5]
                \begin{figure}[!htp]
                        \centering
%                        \includegraphics[width=0.9\linewidth]
%                                {./img/diagrama.png}
                        \caption{\small Um diagrama de experimento}
                        \label{DIAGRAMA}
                \end{figure}
                \lipsum[6]

% Resultados e Discussões
        \section{Resultados e Discussões}
                \lipsum[4]
                \begin{figure}[!htp]
                        \centering
%                        \includegraphics[width=0.9\linewidth]
%                                {./img/grafico.png}
                        \caption{\small Um gráfico.}
                        \label{GRAFICO}
                \end{figure}

                \lipsum[7]

                \begin{figure*}[!htp]
                        \centering
%                       \includegraphics[width=0.9\linewidth]
%                               {./img/diagram2.png}
                        \caption{\small Um diagrama maior.}
                        \label{DIAGRAMA_2}
                \end{figure*}

                \lipsum[10]

                \begin{table}[!htp]
                        \begin{tabular}{m{0.225\linewidth}
                                        m{0.225\linewidth}
                                        m{0.225\linewidth} c}
                                        \toprule
                                % Nome das Colunas
                                Grandeza &
                                \hbox{Valor} Teórico &
                                \hbox{Valor} Prático & Desvio\\
                                \midrule

                                % Linha 1
                                $f$ (\SI{}{\hertz}) &
                                60,00 &
                                60,10 &
                                0,17\% \\

                                % Linha 2
                                $V_{L}$ (\SI{}{\volt}) &
                                30,00 &
                                29,28 &
                                2,42\% \\

                                % Linha 3
                                $I_{L}$ (\SI{}{m\ampere}) &
                                \SI{300,0}{} &
                                \SI{290,0}{} &
                                3,42\% \\

                                % Linha 4
                                $\phi$ (\SI{}{\radian}) &
                                0.00 &
                                0,00 &
                                0,00\% \\ \hline
                        \end{tabular}
                        \caption{\small Uma tabela}
                        \label{REFERENCIA_TABELA}
                \end{table}

                \lipsum[8]
                \cite{NOME_DE_REFERENCIA_LIVRO}

% Conclusões
        \section{Conclusões}
                \lipsum[9-10]
% Bibliografia
        \printbibliography
\end{document}
